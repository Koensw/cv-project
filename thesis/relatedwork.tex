%% ----------------------------------------------------------------------------
% BIWI SA/MA thesis template
%
% Created 09/29/2006 by Andreas Ess
% Extended 13/02/2009 by Jan Lesniak - jlesniak@vision.ee.ethz.ch
%% ----------------------------------------------------------------------------
\newpage
\chapter{Related Work}
\label{ch:related_work}
Unsupervised learning has been studied extensively over the past decades. Before the arrival of CNNs handcrafted features like SIFT, HOG and SURF have been used for classification and detection~\cite{lee2017}\needref. With deep learning visual representations can be extracted directly from data, but a network often requires millions of labeled images for the supervised learning. In a typical way of working CNNs are trained on ImageNet~\cite{deng2009} and then afterwards fine-tuned for a more specific application with limited sources of supervision. 

There are various approaches to circumvent the problem with building supervisised datasets, learning in an unsupervised way instead. In reconstruction-based learning a network tries to learn representation that can reproduce itself, in self-supervised learning context is exploited as supervisory signal.

\section{Reconstruction-based learning}
In a more general sense, we can think of a good image representation as the latent variables of a generative model, but directly inferring this structure is generally intracable~\cite{doersch2015}. One well-researched way to approximate the model is using restricted Boltzmann machines, an generative stochastic artificial neural network trained in various forward and backward passes. In a restricted Boltmann machine there are hidden units that are activated by a stochastic weighted model over the input nodes, and those activations are passed backward in an attempt to reconstruct the input, learning the weights by minimizing reconstruction error~\cite{smolensky1986}. 

Another interesting approach, which works by similar means, is auto-encoding, where the input image also functions as the output of the network. To build useful features in the hidden layers and thus to prevent learning the identity function, images are corrupted with the task of the network being to denoise and reconstruct the original image. Often a sparsity penalty is added to allow the technique to be applied in deep networks. It was shown that this approach makes it possible to make robust human body detectors without lableling, but using full-sized images such a network required up to a million CPU hours to learn some valuable representations~\cite{le2013}.  

\section{Self-supervised learning}
Self-supervised learning is another novel unsupervied learning paradigm that exploits different labelings that are freely available besides or within visual data. Those labelings are usually of limited direct interest to estimate during test time. Instead those sources are often processed as a pretext task with the goal of learning general features that can later be remodelled for vision tasks such as object detection and semantic segmentation\needref. To achieve this, the networks used by those proposed tasks have a common convolutional backbone pipeline extracting features for object detection. Common choices for this multi-layer detection architecture include AlexNet~\cite{krizhevsky2012} (and the closely related CaffeNet~\cite{jia2014}), VGC~\cite{simonyan2014}, GoogleNet~\cite{szegedy2015} and ResNet~\cite{he2016}. For segmantic segmentation the pre-trained architectures can be integrated into a complete convolutational classification framework like the state-of-the-art Mask R-CNN~\cite{he2017} with the weights initialized to those from the self-supervised tasks. Finally smaller supervised datasets with full segmentations can then be used to train the network for classification and segmentation.

Recent work has investigated different sources of intrinsic information within image and video data that can be exploited for self-supervision. Most works have focused on the following main origins of context: 
\begin{itemize}
\item Spatial structure
\item Spatioemporal coherence 
\item Auxilliary data mostly in the form of egomotion.
\end{itemize}

Doersch et al.~\cite{doersch2015} formulate a self-supervised task to learn spatial context in images. The image is first split in an 3x3 grid, and the patch in the middle of the image is taken together with any of the other 8 tiles. Then the network is trained to learn to predict the correct location of the second tile relative to the first. This work was later extended to observe all the tiles at the same time by shuffling the patches and learning to solve the created jigsaw puzzle~\cite{noroozi2016}. Another related technique was put forward by Wang \& Cupta~\cite{wang2015} that tries to exploit different views of the same object and use known similarity as a training loss. It does this by mining two images of the same object, found using unsupervised KLT-tracking with SURF features in video data, and another totally unrelated random image. The setup uses a Siamese network for the three images with the final loss penalizing differences in the visual representation of the same object and on the opposite side similarities in comparison with the random image.

This last approach makes use of temporal coherence, but is ultimately an spatial learning method requiring significant pre-processing to create the similarity task. Misra et al.~\cite{misra2016} opt for an easier way to learn directly from the temporal coherence. They mine sequences of three frames from videos and shuffle these in a random order, using the fact if the permutation is sorted or not as supervision source. This approach was expanded later to sequences of longer length with the task of the neural network to learn the full permutation by Lee et al.~\cite{lee2017}. Their network uses a CNN in a Siamese structure for independent feature extraction in the backbone network, followed by a pair-wise feature extraction and a fully-connected layer for the final order prediction. In another variation, the network is fed $N$ correctly ordered tuples and one other in the wrong order. The network can then try to learn to identify the wrongly ordered tuple in all $N+1$ tuple inputs~\cite{fernando2017}. Both approaches makes the learning formulation more complicated, allowing learning richer representations\needref. 

A final interesting form of supervision is ego-motion, which include the information about movement as recorded by other sensors. Living species also leverage self-generated movement in concert with visual feedback for proper perceptual development. Inspired by this concept, neural networks should also be able to benefit from information from motoral activity. Agrawal et al.~\cite{agrawal2015} directly use information about camera transformations, their agent optimizes it visual representations by minimizing the error between the egomotion obtained from the motor system and egomotion predicted using the network only. The two image frames go through a Siamese base architecture followed by fully-connected layers to estimate the transformation. In a similar work Jayaraman \& Grauman learn a equivariant feature space using ego-motion by mapping the feature space between two frames and calculating the loss based on the difference between the features.

\section{3D data in CNNs}
...
