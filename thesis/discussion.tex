%% ----------------------------------------------------------------------------
% BIWI SA/MA thesis template
%
% Created 09/29/2006 by Andreas Ess
% Extended 13/02/2009 by Jan Lesniak - jlesniak@vision.ee.ethz.ch
%% ----------------------------------------------------------------------------
\newpage
\chapter{Discussion}
\label{ch:discussion}
In the previous section it has been shown that CNNs are very well capable of learning the ordering task, both the simple 3-input binary task reaching maximum 82,84\% as well as the 4-input 24-output permutation version with an top-1 accuracy of 44,42\%. While the networs are definitely capable learning this task, it interesting to find out what representation the networks actually primarily learn. As the task does not (directly) require the detection of any object it is not expected that the network only learns to detect a particular class of objects. But what does it learn instead?

% \begin{itemize}
%  \item \textit{What do your results mean?} Here you discuss, but you do not recapitulate results. Describe principles, relationships and generalizations shown. Also, mention inconsistencies or exceptions you found.
%  \item \textit{How do your results relate to other's work?} Show how your work agrees or disagrees with other's work. Here you can rely on the information you presented in the ``related work'' section.
%  \item \textit{What are implications and applications of your work?} State how your methods may be applied and what implications might be. 
% \end{itemize}
% 
% \noindent Make sure that introduction/related work and the discussion section act as a pair, i.e. ``be sure the discussion section answers what the introduction section asked''. 
